\documentclass[aspectratio=169]{beamer}
\usepackage{amsmath}
\usepackage{mathpazo, times}
\usepackage{tikz}

% https://tex.stackexchange.com/questions/353100/is-there-a-dotted-box-placeholder-symbol
\def\placeholder{\tikz\node[draw=black,dashed,very thick] {\vphantom{.}};}

\beamertemplatenavigationsymbolsempty

\begin{document}

\fontsize{22}{22}\selectfont

\begin{frame}[t]
\begin{equation*}
\begin{array}{l@{\;=\;}l}
\mathit{poly}(x) & 3x^2 + 4 \\ \pause
 & (\placeholder + 4) \cdot (3 \times \placeholder) \cdot \placeholder^2
\end{array}
\end{equation*}

\pause

\begin{equation*}
(f \cdot g)(x) \pause = f(g(x))
\end{equation*}

\end{frame}

\begin{frame}

\begin{equation*}
(f \cdot g)(x) = f(g(x))
\end{equation*}

\pause

\begin{equation*}
\begin{array}{l@{\;=\;}l}
(f \cdot (g \cdot h)) (x) & \pause f((g \cdot h)(x)) \\
& \pause f(g(h(x)) \\
\end{array}
\end{equation*}

\pause

\begin{equation*}
\begin{array}{l@{\;=\;}l}
((f \cdot g) \cdot h)(x) & \pause (f \cdot g)(h(x)) \\
& \pause f(g(h(x)) \\
\end{array}
\end{equation*}

\pause

\begin{equation*}
f \cdot (g \cdot h) = (f \cdot g) \cdot h = f \cdot g \cdot h
\end{equation*}

\end{frame}

\begin{frame}

%\begin{equation*}
%(f \cdot g)(x) = f(g(x))
%\end{equation*}
%
%\pause

\begin{equation*}
\mathtt{id}(x) = x
\end{equation*}

\pause

\begin{equation*}
\begin{array}{l@{\;=\;}l}
(f \cdot \mathtt{id})(x) & \pause f(\mathtt{id}(x)) \\
& \pause f(x)
\end{array}
\end{equation*}

\pause

\begin{equation*}
\begin{array}{l@{\;=\;}l}
(\mathtt{id} \cdot f)(x) & \pause \mathtt{id}(f(x)) \\
& \pause f(x)
\end{array}
\end{equation*}

\pause

\begin{equation*}
f \cdot \mathtt{id} = \mathtt{id} \cdot f = f
\end{equation*}

\end{frame}

\begin{frame}

\begin{equation*}
(f \cdot g)(x) = f(g(x))
\end{equation*}

\vspace{1em}

\begin{equation*}
\begin{array}{l@{\;=\;}l@{\;=\;}l}
f \cdot (g \cdot h) & (f \cdot g) \cdot h & f \cdot g \cdot h \\[1em]
f \cdot \mathtt{id} & \mathtt{id} \cdot f & f
\end{array}
\end{equation*}

\end{frame}

\end{document}
